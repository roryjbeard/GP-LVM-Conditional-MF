\documentclass[12pt]{article}
\usepackage{amsmath}
\usepackage{amsfonts}
\usepackage{amssymb}
\usepackage[top=20mm, bottom=20mm, left=30mm, right=30mm]{geometry}
\usepackage{soul}
\usepackage{bbm}
%Gummi|061|=)
\usepackage{setspace}
\usepackage[usenames, dvipsnames]{color}

\newcommand{\chris}[1]{\textcolor{ForestGreen}{#1}}
\newcommand{\Tau}{\mathcal{T}}
\newcommand{\Kff}{K_{X_fX_f}}
\newcommand{\Kuu}{K_{X_uX_u}}
\newcommand{\Kuf}{K_{X_uX_f}}
\newcommand{\Kfu}{K_{X_fX_u}}
\newcommand{\Ex}{\mathbb{E}}
\newcommand{\No}{\mathcal{N}}
\newcommand{\chol}{\mathrm{chol}}
\doublespacing


\title{\textbf{Derivations and Equations}}
\date{}
\begin{document}

\maketitle


\section{Deriving the bound} % (fold)
\label{sec:derive_bound}
The standard variational formulation is
%
\begin{align}
    \log p(y) &= \int q(z)\log\frac{p(y|z)p(z)}{q(z)}\frac{q(z)}{p(z|y)} dz\notag\\
    &= \underbrace{\mathbb{E}_{q(z)}\big[ \log p(y|z)p(z) - \log q(z) \big]}_{\mathcal{L}_1} + \text{KL}\big[ q(z)\Vert p(z|y) \big]\label{eq:logpy1}
\end{align}
%
The GP-LVM-CMF variational joint in the augmented latent space is:
%
\begin{equation}
    q(z,f,u,X) = \prod_{i=1}^N \{q(z_i|f(X_i))q(f(X_i)|u)\}q(u)q(X)
\end{equation}
%
which induces an intractable variational marginal
%
\begin{equation}
    q(z) = \int \prod_{i=1}^N \{q(z_i|f(X_i))q(f(X_i)|u)\}q(u)q(X) df du dX.
\end{equation}
%
However we can apply the variational formulation again, noting that $q(z) = \frac{q(z|f,u,X)q(f,u,X)}{q(f,u,X|z)}$, and integrating \ref{eq:logpy1} wrt the auxiliary variables, to obtain a further lower bound:
\begin{align}
    \log p(y) &= \int q(z|f,u,X)q(f,u,X)\log \frac{p(y|z)p(z)r(f,u,X|z)}{q(z|f,u,X)q(f,u,X)}\frac{q(f,u,X|z)}{r(f,u,X|z)}\frac{q(z)}{p(z|y)} df du dX dz\notag\\
    &= \mathbb{E}_{q(z,f,u,X)}\big[ \log p(y|z)p(z)r(f,u,X|z) - \log q(z|f,X)q(f,u,X) \big]\label{eq:L_aux}\\
    &\qquad+ \mathbb{E}_q(z)\big[ \text{KL}[q(f,u,X|z)\Vert r(f,u,X|z)] \big] + \text{KL}\big[ q(z)\Vert p(z|y) \big]
\end{align}
%
in which the new auxiliary lower bound, $\mathcal{L}_{aux}$ is given by the expression \ref{eq:L_aux}, and where we have introduced the auxiliary distribution $r(f,u,X|z)$ which serves to approximate the variational posterior, $q(f,u,X|z)$, of the auxiliary variables conditioned on the latent variables.\\
We may re-express $\mathcal{L}_{aux}$ in a way which makes use of the analytical expression for the K-L divergence between two Gaussians, $q(f,u,x)$ and $r(f,u,X|z)$ and, in the case that the prior of the generative model, $p(z)$, is also Gaussian distributed - as is the case for the continuous latent variable MLP model we'll consider first - then the bound contains a second Gaussian KL term:
%
\begin{equation}
    \mathcal{L}_{aux} = \underbrace{\mathbb{E}_{q(z)}\big[ \log p(y|z) \big]}_{A} - \underbrace{\mathbb{E}_{q(f,u,X)}\big[ \text{KL}[q(z|f,X)\Vert p(z)] \big]}_{B} - \underbrace{\mathbb{E}_{q(z)}\big[ \text{KL}[q(f,u,X)\Vert r(f,u,X|z)] \big]}_{C}.
\end{equation}
%
%
\section{Expressions}
\subsection{Generative model}
%
\begin{equation}
    p(z) = \mathcal{N}(z\mid 0, I)
\end{equation}
%
\subsubsection{Continuous data $\implies$ Gaussian MLP likelihood}
%
\begin{align}
    \log p(y|z) &= \log\mathcal{N}(x\mid \mu, \sigma^2 I)\\
    \text{where } \mu &= W_2 h + b_2\\
    \log\sigma^2 &= W_3 h + b_3\\
    h &= \tanh(W_1 z + b_1)
\end{align}
%
\subsubsection{Discrete data $\implies$ Bernoulli MLP likelihood}
%
\begin{align}
    \log p(y|z) &= \sum_{p=1}^P y_p\log h_p + (1-y_p)\log (1-h_p)\\
    \text{where } h &= f_{\sigma}(W_2 \tanh(W_1 z + b_1) + b2)
\end{align}
\subsection{Variational model}
%
\begin{align}
q(X) &= \prod_{i=1}^N \mathcal{N}(x_{:,n}\mid 0, I_R) = \prod_{n=1}^N \mathcal{N}(x_{n,:}\mid 0,1)
     \\
    q(u) &= \prod_{m=1}^M \mathcal{N}(u_{:,m}\mid \kappa_{:,m}, \Kuu)\\
    q(f(X)|u) &= \mathcal{N}(f(X)\mid \Kfu\Kuu^{-1}u, \Kff-\Kuf\Kuu^{-1}\Kuf)\\
    q(z|f(X)) &= \prod_{i=1}^N \mathcal{N}(z_i\mid f(X_i), \sigma^2)
\end{align}

\chris{I think the variational distributions need to have variational parameters. Also there is a single guassian process, right? If so q(u) is just one gaussian, not a product of gaussians}
\begin{align}
    q(X) &= \prod_{i=1}^N \mathcal{N}(X_i\mid \phi_i, \Phi_i)\\
    q(u) &= \mathcal{N}(u \mid \kappa, K_{X_uX_u})\\
\end{align}

\subsection{Auxiliary model}
%
\begin{align}
    r(f,u,X|z) &= q(f(X)|u,X)r(u,X|z)\\
    r(u,X|z) &= r(u| z) \prod_i r(X_i|z) = \mathcal{N}(u; \upsilon, \Upsilon ) \prod_i \mathcal{N}( X_i, \tau_i, \Tau_i )
\end{align}



\subsection{}

\begin{align}
\Ex_q(z) \mathcal{L}(z) &= \Ex_q(z,f,u,X) \mathcal{L}(z,f,u,X) \\
&= \Ex_{\No(\eta;0,1)\No(\xi;0,1)\No(\alpha;0,1)\prod\No(\beta_i,0,1)} \mathcal{L}_1(\eta,\xi,\alpha,\beta) \\
\mathcal{L}_1(\eta,\xi,\alpha,\beta) &=  \mathcal{L}(z,f,u,X)
\end{align}

where  $z = f + \mathbf{I}\sigma\eta $

where $\Sigma = \Kff - \Kfu\Kuu^{-1}\Kuf$

where $\mu   = \Kfu\Kuu^{-1}u$

where $f = \mu + \chol(\Sigma)\xi $

where $u = \upsilon + \chol(\Upsilon)\alpha$

where $X_i = \tau_i + \chol(\Tau_i)\beta_i$








\end{document}
